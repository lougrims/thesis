\documentclass{iitthesis}

%   \documentclass[draft]{iitthesis}

\usepackage[dvips]{graphicx}    % This package is used for Figures
\usepackage{rotating}           % This package is used for landscape mode.
\usepackage{epsfig}
\usepackage{subfigure}          % These two packages, epsfig and subfigure, are used for creating subplots.
% Packages are explained in the Help document.


\begin{document}
%%fakesection{TITLES AND INDEXES)
%%% Declarations for Title Page %%%
\title{How to Write Theses\\
  With Two Line Titles}
\author{Lou Grimaud}
\degree{Master of Science}
\dept{Mechanical, Materials, and Aerospace Engineering}
\date{May 2014}
%\copyrightnoticetrue      % crate copyright page or not
\maketitle                % create title and copyright pages


\prelimpages         % Settings of preliminary pages are done with \prelimpages command


%%%  Acknowledgement %%%
\begin{acknowledgement}     % acknowledgement environment, this is optional
\par  This dissertation could not have been written without Dr. X
who not only served as my supervisor but also encouraged and
challenged me throughout my academic program. He and the other
faculty members, Dr. Y and Dr. Z, guided me through the
dissertation process, never accepting less than my best efforts. I
thank them all.\\ \\ (Don't copy this sample text. Write your own
acknowledgement.)
%\input{acknowledgement.tex} % you need a separate acknowledgement.tex file to include it.
\end{acknowledgement}


% Table of Contents
\tableofcontents
\clearpage

% List of Tables
\listoftables

\clearpage

%List of Figures
\listoffigures

\clearpage

%List of Symbols(optional)

\listofsymbols
 \SymbolDefinition{$\beta$}{probability of non-detecting bad data}
 \SymbolDefinition{$\delta$}{Transition Coefficient Constant for the Design of Linear-Phase FIR Filters}
 \SymbolDefinition{$\zeta$}{Reflection Coefficient Parameter}


 \clearpage

%%% END OF INDEXES %%%
%%%%%%%%%%%%%%%%%%%%%%%%%%%%%%%%%%%%%%%%%%%%%%%%%%%%%%%%%%%%%%%%%%%%%%%%%%%%%%%%%%%%%%%

%%% Abstract %%%
\begin{abstract}           % abstract environment, this is optional
 \par The purpose of this thesis is to show how micro unmanned aerial vehicles can extract energy from natural wind gusts and how this energy extraction is affected by the effects of unsteady aerodynamics.

\par The trajectory of a small UAV flying through wind gusts is simulated with a two degrees of freedom model.
The non-dimensional model is set to include vertical and horizontal gusts of varying amplitudes and durations.
From this model an optimization routine is performed in order to obtain the minimum gust amplitude needed to get a neutral energy trajectory.
With these results, it is shown that neutral energy flight is possible through gusts speeds only 10 to 30\% of the flying speed of the aircraft .
Analysis of the results shows that the lift coefficient has to be changed very rapidly in order to perform these maneuvers in short duration gusts. 
Moreover high lift values are often required. 

\par To achieve this kind of rapid changes in the lift and drag forces, fast variations of the angle of attack are needed.
The high lift values also requires high angles of attacks that are likely to cause separation of the flow over the airfoil.
These fast variations at high angle of attack are shown to cause unsteady non linear aerodynamic responses.
Traditional CFD simulations are far too computationally expensive to be implemented into the optimization routine.
To solve this issue a low order model based on a paper by Goman and Khrabrov \cite{GK} is developed and validated against experimental results.
This model produces accurate predictions of the lift and drag coefficients for a wide range of angle of attack and for different type of pitch inputs.

\par With this light model the influence of the unsteady aerodynamics on the energy extraction problem are highlighted.
The main difference with quasi-steady aerodynamics model was found to be for gusts at a reduced frequency faster than k of 0.07.
Around these values the potential performances are improved by introducing the unsteady model.
The trajectories obtained include more violent changes in angle of attacks in order to take full advantages of the unsteady effects.



  %you need a separate abstract.tex file to include it.
\end{abstract}


\textpages     % Settings of text-pages are done with \textpages command

% Chapters are created with \Chapter{title} command
\Chapter{INTRODUCTION}

\input{Chapters/introduction.tex}
\Chapter{Energy extraction optimization}

\Section{2 DoF model}

\Subsection{Non-dimensional equations of motion}

\Subsection{Lift and drag models}

\Section{Optimization process, cost function and constraints}

\Subsection{General consideration on optimization}
% Get some stuff from the MM544 class

\Subsection{Cost function and constraints formulation}

\Section{Results}

\Subsection{Horizontal gusts}

\Subsection{Vertical gusts}

\Subsection{Combined gusts}



\Subsection{Preliminary analysis of the results}

\Chapter{Active Flow Control}

\Section{Experimental setup}

\Subsection{wind tunnel}

\Subsection{Leading edge piezoelectric actuators}

\par{NACA 009 airfoil with leading edge synthetic jets}

\par{Driver for the piezoelectric elements}

\Section{Single pulse response}

\Subsection{Pulse design}
% high frequency to frive the piezos (find paper for that)

\Subsection{Influence of the pulse duration}

\Subsection{Effect of pulse duration}

\Subsection{Effect of pulse separation}

\Subsection{Effect of pulse amplitude}


\Section{Pulse train design}

\Subsection{Design constraints}
% bandwidth consideration

\Subsection{Chosen parameter}

\Subsection{Square input example}




\Section{Experimental results for different input signals}

\Subsection{continual and square input signals}

\Subsection{triangle input signal}

\Section{System identification}
% discuss about dead time, lift reversal, asymmetry

\Chapter{Modelization of the Lift coefficient under unsteady pitching motion)}

%servo drive and stuff

\Section{Theoretical background of the GK model}

\Section{GK model adaptation}

\Subsection{Steady list and stalling behavior}

\Subsection{State variable approximation}

\Section{Experimental Setup}

\Section{Model validation}
% put both sinusoidal and ``random'' motion

\Chapter{Controller design and validation}

\Section{Proof of concept for combining AFC with pitching}

\Subsection{Open loop experimental results}

\Subsection{Comparison with models output}

\Section{Controller structure}

\Subsection{AFC transfer function inverse}

\Subsection{Controller input and objective discussion}

\Subsection{Controller block choice}

\Section{Results}

\Subsection{Periodic pitching motion}

\Subsection{Limited bandwidth random pitching motion}

\Section{Discussion of the results}

\Subsection{Bandwidth limitation}

\Subsection{Precision issues}

\Section{Possible improvements and feedback implementation}

\Chapter{CONCLUSION}

% \Section{Low order model for unsteady flow}

\par The Goman and Khrabrov model shows good agreement in both shape and magnitude for the lift and drag of a pitching airfoil in a wide range of conditions.
It is fast and light and can be adjusted to a new airfoil (or even a whole aircraft) without requiring extensive experimental studies.
In fact in theory only a static map and one unsteady case should be enough to obtain the whole model.

\par The work in this thesis has proven that the drag can also be easily modeled and there is little doubt that the moment coefficient $C_m$ could be modeled too.

\par Implementing this model is simple enough that it can be used in computationally heavy applications such as optimization problems or real time applications.
It is however non-linear and as such isn't easily invertible or analyzable with traditional control system methods.

\Section{Energy extraction optimization}

\par From the results presented in this thesis it can be seen that energy extraction can be performed for complex temporal wind gusts with vertical and horizontal components.
In these gusts overall neutral energy trajectories are possible and require less and less wind gust amplitude the shorter the gust is.
For short gust ($T_g \le 3$) high angles of attack are needed for maximum performance, which can lead to flow separation.

\par The introduction of an unsteady aerodynamic model is possible and proves necessary when gusts are very short ($T_g \le 0.7$ or $k \ge 0.05)$).
At this point unsteady effects such as lag in the flow separation can be observed and even taken advantage of.
While the maneuvers required to obtain such trajectories are unlikely to be realistically done on aircrafts, these results could be exploited in vertical wind turbine pitch optimization for example.
The results for gusts shorter than $T_g=0.2$ ($k=0.175$) are less obvious to interpret and will require further study, perhaps with a problem more representative of the system operating at such frequencies.

\par While these results shows the optimal trajectory found by the algorithm it is important to keep in mind that the algorithm optimize the trajectory globally over the whole period.
This means that it can take preemptive actions because it ``knows'' what will happen later in time.
A real controller flying in an environment where the wind gust shape can't be predicted would not be able to reach the same energy extraction performances.

\Section{Possible improvements and additional work}

\par One of the weakness of the optimization with the GK model is that it does not account for the unsteady effects due to the surging and gusting components of the relative wind.
I believe that the GK model could be modified to include these effects but a complete experimental campaign would be needed.

\par The ``staircase'' effect described in section \ref{sub:staircase} could be mitigated by introducing a more realistic model for the aircraft dynamics that include the moment of inertia.
Going even further, a basic elevator model could be implemented.

\par Finally replacing the temporal wind profile by a spatial one would allow for the use of more realistic and complex wind fields.



\Section{Summary}

This was just to create a sample section...

\clearpage


%
% APPENDIX
%

% Do the settings of appendices with \appendix command
\appendix

% Then create each appendix using
% \Appendix{title_of_appendix} command

\Appendix{Table of Transition Coefficients for the Design of
Linear-Phase FIR Filters}
Your Appendix will go here !

 \moretox

  \Appendix{Name of your Second
Appendix}

Your second appendix text....

\Appendix{Name of your Third Appendix}

Your third appendix text....
%
% BIBLIOGRAPHY
%
% you have two options: 1) create bibliography manually,
% 2) create bibliography automatically. See BibliographyHelp.pdf file for details.


\bibliographystyle{plain}
\bibliography{mybib}


\end{document}  % end of document
