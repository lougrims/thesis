\Section{Motivation} 

\Subsection{Dynamic soaring} \label{subsec:dynsoar}

\par The main challenge electric small size unmanned aerial vehicle is the autonomy.
Battery energy density is limited and can rapidly become a important part of the weight of vehicle.
Since most of the energy is used by the electric engine for propulsion. 
With the progress in autonomous control software successful attempt have been made by Allen \cite{flight_test_soaring_NASA} and Edwards \cite{flight_test_soaring_NCU} to extract energy from natural updraft.
These experiment has shown that a UAV can take advantage of localized vertical gusts naturally produced by thermal convection effects.

\par However, within an urban environment, such as the one mini and micro-UAV are dedicated to, the gust's profile is vastly different. [INSERT REF FROM NEXT DOORS LAB !!!]. 
Wind blowing through an group of building produce turbulent conditions with both vertical and horizontal vortexes.

\par \emph{Figure from, PIV of the building array}

\par In flow fields such as this the gusts encountered are both faster and more arguably more complex than the ones due to thermal convection.
The resulting effect is that the optimal trajectory through such gust profile is different from the one shown developed previously by XXXXXXXXXXXXXX.
 
\Subsection{Active flow control system}

The dynamic soaring maneuvers required for energy extraction in environments described in \ref{subsec:dynsoar} necessity rapid variations of the lift coefficient. If these variations are performed by pitch change a number of unsteady effect are introduced and can produce some unexpected lift behaviors.
