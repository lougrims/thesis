\Section{2 DoF model}

\Subsection{Non-dimensional equations of motion}

\par The model chosen for this simulation is a simple two degree of freedom, two dimension, point mass model. 
The aircraft is assumed to be a glider to simplify the optimization routine. 
With such assumption the equations of motion in the ground reference frame is :

\begin{equation}
	\begin{array}[c]{c}
		\ddot{x}= -L' \cdot sin(\gamma) + D' \cdot cos(\gamma) \\ 
		\ddot{z}= L' \cdot cos(\gamma) - D' \cdot sin(\gamma) - m \cdot g
	\end{array}
	\label{eqn:eqm}
\end{equation}

% should I include a figure with the reference frame, and angle definition?

\par The lift and drag are defined are: 

\begin{equation}
	\begin{array}[c]{c}
		L'= \frac{1}{2} \rho V^2 C_l \\ 
		D'= \frac{1}{2} \rho V^2 C_d 
	\end{array}
	\label{eqn:Cl_def}
\end{equation}

\par With $V$ being the relative wind for the vehicle.
% do I need to include this kind basic stuff?

\par Since this simulation is mainly concerned with Newtonian physics (rather than fluid phenomenons) the usual fluid dynamics non-dimensional variables make little sens.
Here the equations are normalized by the optimal glide speed and $g$, the gravitational acceleration.
This is more representative of the performances of the aircraft.

\par Following Lissaman's \cite{lissaman2005wind} implementation of the equation of motion we define $V^*$ the optimal glide speed for the aircraft. This speed is achieved at the optimal lift to drag ratio of the aircraft.
With $C_l^*$ and $C_d^*$ the angle of attack for the maximum lift to drag ratio and $\gamma$ the pitch angle with respect to the horizon the optimal glide speed is:

\begin{equation}
	\begin{array}[c]{c}
		\gamma^*= - atan(\frac{C_l^*}{C_d^*}) \\
		V^* = \sqrt{\frac{2mg}{\rho S (C_l^* cos(\gamma^*) - C_d^* sin(\gamma^*)}}
	\end{array}
	\label{eqn:glide_speed}
\end{equation}

\par From this we define $U$ and $W$ the non dimensional horizontal and vertical speed in the inertial reference frame.

\begin{equation}
	\begin{array}[c]{c}
		U = \frac{\dot{x}}{V^*} \\
		V= \frac{\dot{z}}{V^*}
	\end{array}
	\label{eqn:non_dim_speed}
\end{equation}
The time is normalized by $g / V^*$.

\par Since the speed is seen as a fraction of the optimal glide speed it makes sens to also normalize the lift and drag coefficients by their corresponding values at the optimal lift to drag ratio.

\begin{equation}
	\begin{array}[c]{c}
		L= \frac{C_l}{C_l^*} \\
		D= \frac{C_d}{C_d^*} 
	\end{array}
	\label{eqn:non_dim_coef}
\end{equation}

\par Finally we introduce Q the dynamic pressure as:

\begin{equation}
	Q = \frac{L'}{MgL} = \frac{\frac{1}{2} \rho V^2 C_l C_l^* }{Mg}
	\label{eqn:dynamic_pressure}
\end{equation}

\par From there the equation of motion \ref{eqn:eqm} can be expressed as:

\begin{equation}
\begin{array}[c]{c}
	\frac{dU}{dT}= -LQ \cdot sin(\gamma) + DQ \cdot cos(\gamma) \\ 
	\frac{dW}{dT}= LQ \cdot cos(\gamma) - DQ \cdot sin(\gamma) - 1
	\end{array}
	\label{eqn:non_dim_eqm}
\end{equation}
\Subsection{Lift and drag models}

\Section{Optimization process, cost function and constraints}

\Subsection{General consideration on optimization}
% Get some stuff from the MM544 class

\Subsection{Cost function and constraints formulation}

\Section{Results}


