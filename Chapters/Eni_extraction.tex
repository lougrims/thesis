\Section{2 DoF model}

\Subsection{Non-dimensional equations of motion}

\par The model chosen for this simulation is a simple two degree of freedom, two dimension, point mass model. 
The aircraft is assumed to be a glider to simplify the optimization routine. 
With such assumption the equations of motion in the ground reference frame is :

\begin{equation}
	\begin{array}[c]{c}
		\ddot{x}= -L' \cdot sin(\gamma) + D' \cdot cos(\gamma) \\ 
		\ddot{z}= L' \cdot cos(\gamma) - D' \cdot sin(\gamma) - m \cdot g
	\end{array}
	\label{eqn:eqm}
\end{equation}

% should I include a figure with the reference frame, and angle definition?

\par The lift and drag are defined are: 

\begin{equation}
	\begin{array}[c]{c}
		L'= \frac{1}{2} \rho V^2 C_l \\ 
		D'= \frac{1}{2} \rho V^2 C_d 
	\end{array}
	\label{eqn:Cl_def}
\end{equation}

\par With $V$ being the relative wind for the vehicle.
% do I need to include this kind basic stuff?

\par Since this simulation is mainly concerned with Newtonian physics (rather than fluid phenomenons) the usual fluid dynamics non-dimensional variables make little sens.
Here the equations are normalized by the optimal glide speed and $g$, the gravitational acceleration.
This is more representative of the performances of the aircraft.

\par Following Lissaman's \cite{lissaman2005wind} implementation of the equation of motion we define $V^*$ the optimal glide speed for the aircraft. This speed is achieved at the optimal lift to drag ratio of the aircraft.
With $C_l^*$ and $C_d^*$ the angle of attack for the maximum lift to drag ratio and $\gamma$ the pitch angle with respect to the horizon the optimal glide speed is:

\begin{equation}
	\begin{array}[c]{c}
		\gamma^*= - atan(\frac{C_l^*}{C_d^*}) \\
		V^* = \sqrt{\frac{2mg}{\rho S (C_l^* cos(\gamma^*) - C_d^* sin(\gamma^*)}}
	\end{array}
	\label{eqn:glide_speed}
\end{equation}

\par From this we define $U$ and $W$ the non dimensional horizontal and vertical speed in the inertial reference frame.

\begin{equation}
	\begin{array}[c]{c}
		U = \frac{\dot{x}}{V^*} \\
		V= \frac{\dot{z}}{V^*}
	\end{array}
	\label{eqn:non_dim_speed}
\end{equation}
The time is normalized by $g / V^*$.

\par Since the speed is seen as a fraction of the optimal glide speed it makes sens to also normalize the lift and drag coefficients by their corresponding values at the optimal lift to drag ratio.

\begin{equation}
	\begin{array}[c]{c}
		L= \frac{C_l}{C_l^*} \\
		D= \frac{C_d}{C_d^*} 
	\end{array}
	\label{eqn:non_dim_coef}
\end{equation}

\par Finally we introduce Q the dynamic pressure as:

\begin{equation}
	Q = \frac{L'}{MgL} = \frac{\frac{1}{2} \rho V^2 C_l C_l^* }{Mg}
	\label{eqn:dynamic_pressure}
\end{equation}

\par From there the equation of motion \ref{eqn:eqm} can be expressed as:

\begin{equation}
\begin{array}[c]{c}
	\frac{dU}{dT}= -LQ \cdot sin(\gamma) + DQ \cdot cos(\gamma) \\ 
	\frac{dW}{dT}= LQ \cdot cos(\gamma) - DQ \cdot sin(\gamma) - 1
	\end{array}
	\label{eqn:non_dim_eqm}
\end{equation}

With 

\begin{equation}
	\gamma = -atan(\frac{W}{U})
	\label{eqn:gamma_def}
\end{equation}

\par Finally the remaining thing to consider is $Q$ the dynamic pressure. If we define the speed of the wind gust as $W_g$ and $U_g$ we can express:

\begin{equation}
	Q = V^2 = (W-W_g)^2 + (U-U_g)^2
	\label{eqn:q_def}
\end{equation}

\par With these definitions we have the basic formulation of our non-dimensional equation of motions, normalized by the performances at the optimal glide trajectory in a calm environment.

\Subsection{Lift and drag models}

\par The normalized equation of motion \ref{eqn:non_dim_eqm} are not accounting for the fluid dynamic part of the flight.
The most important factor for glide performance is the lift to drag ratio. 
In his paper, Lissaman \cite{Lissaman2007neutral} is using a relatively simple quadratic model for the relationship between lift and drag:

\begin{equation}
	D=\frac{L}{2G}(1+L^2)
	\label{eqn:Lissaman_G}
\end{equation}

[CHECK THIS EQUATION !!!!!]

\par This simple model work relatively well for simple airfoil but is inadequate for more complex shapes.
For various reasons, small UAVs tend to have non classical designs such as blended wing body or flying wing shapes.
The advantages of these designs reside in bigger space available for the payload while keeping the drag low.
However the flying wing airfoil profiles have very different lift to drag characteristics compared to more classical airfoils.

\par To get the lift to drag curve of a typical flying wing a typical flying wing shape is tested in the XFLR5 software.

\par [FIGURE OF THE UAV !!!]

\par This software perform flow simulation over the aircraft with a panel based method. 
The simulation is performed on a 


\Section{Optimization process, cost function and constraints}

\Subsection{General consideration on optimization}
% Get some stuff from the MM544 class

\Subsection{Cost function and constraints formulation}

\Section{Results}


